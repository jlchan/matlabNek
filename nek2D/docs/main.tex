%&pdflatex
\documentclass[final,leqno]{siamltex}
%\documentclass[10pt,onecolumn]{article}
\usepackage[top=2cm,bottom=3cm,left=3.5cm,right=3.5cm]{geometry}
\usepackage[utf8]{inputenc}
\usepackage{amsmath,amssymb,amsfonts,mathrsfs}
\let\proof\relax 
\let\endproof\relax
\usepackage{listings}
\usepackage{array}
\usepackage{mathtools}
\usepackage{dsfont}
\usepackage{graphicx}
\usepackage{pdfpages}
\usepackage[textsize=footnotesize,color=green]{todonotes}
\usepackage{algorithm, algorithmic}
\usepackage{array}
\usepackage{bm}
\usepackage{tikz}
\usepackage{subfigure}
\usepackage[normalem]{ulem}

%\usepackage{lineno}
%\pagewiselinenumbers
%\usepackage{uselinenofix}


\newcommand{\bs}[1]{\boldsymbol{#1}}

\newcommand{\equaldef}{\stackrel{\mathrm{def}}{=}}

\newcommand{\tablab}[1]{\label{tab:#1}}
\newcommand{\tabref}[1]{Table~\ref{tab:#1}}

\newcommand{\theolab}[1]{\label{theo:#1}}
\newcommand{\theoref}[1]{\ref{theo:#1}}
\newcommand{\eqnlab}[1]{\label{eq:#1}}
\newcommand{\eqnref}[1]{\eqref{eq:#1}}
\newcommand{\seclab}[1]{\label{sec:#1}}
\newcommand{\secref}[1]{\ref{sec:#1}}
\newcommand{\lemlab}[1]{\label{lem:#1}}
\newcommand{\lemref}[1]{\ref{lem:#1}}

\newcommand{\mb}[1]{\mathbf{#1}}
\newcommand{\mbb}[1]{\mathbb{#1}}
\newcommand{\mc}[1]{\mathcal{#1}}
\newcommand{\nor}[1]{\left\| #1 \right\|}
\newcommand{\snor}[1]{\left| #1 \right|}
\newcommand{\LRp}[1]{\left( #1 \right)}
\newcommand{\LRs}[1]{\left[ #1 \right]}
\newcommand{\LRa}[1]{\left\langle #1 \right\rangle}
\newcommand{\LRc}[1]{\left\{ #1 \right\}}
\newcommand{\LRb}[1]{\left| #1 \right|}

\newcommand{\tanbui}[2]{\textcolor{blue}{\sout{#1}} \textcolor{red}{#2}}
\newcommand{\Grad} {\ensuremath{\nabla}}
\newcommand{\Div} {\ensuremath{\nabla\cdot}}
\newcommand{\Nel} {\ensuremath{{N^\text{el}}}}
\newcommand{\jump}[1] {\ensuremath{\LRs{\!\left[#1\right]\!}}}
\newcommand{\uh}{\widehat{u}}
\newcommand{\fnh}{\widehat{f}_n}
\renewcommand{\L}{L^2\LRp{\Omega}}
\newcommand{\pO}{\partial\Omega}
\newcommand{\Gh}{\Gamma_h}
\newcommand{\Gm}{\Gamma_{-}}
\newcommand{\Gp}{\Gamma_{+}}
\newcommand{\Go}{\Gamma_0}
\newcommand{\Oh}{\Omega_h}

\newcommand{\eval}[2][\right]{\relax
  \ifx#1\right\relax \left.\fi#2#1\rvert}

\def\etal{{\it et al.~}}

\newcommand{\vect}[1]{\ensuremath\boldsymbol{#1}}
\newcommand{\tensor}[1]{\underline{\vect{#1}}}
\newcommand{\del}{\Delta}
\let\grad\relax
\newcommand{\grad}{\nabla}
\newcommand{\curl}{\grad \times}
\renewcommand{\div}{\grad \cdot}
\newcommand{\ip}[1]{\left\langle #1 \right\rangle}
\newcommand{\eip}[1]{a\left( #1 \right)}
\newcommand{\pd}[2]{\frac{\partial#1}{\partial#2}}
\newcommand{\pdd}[2]{\frac{\partial^2#1}{\partial#2^2}}

\newcommand{\circone}{\ding{192}}
\newcommand{\circtwo}{\ding{193}}
\newcommand{\circthree}{\ding{194}}
\newcommand{\circfour}{\ding{195}}
\newcommand{\circfive}{\ding{196}}

\def\arr#1#2#3#4{\left[
\begin{array}{cc}
#1 & #2\\
#3 & #4\\
\end{array}
\right]}
\def\vecttwo#1#2{\left[
\begin{array}{c}
#1\\
#2\\
\end{array}
\right]}
\def\vectthree#1#2#3{\left[
\begin{array}{c}
#1\\
#2\\
#3\\
\end{array}
\right]}
\def\vectfour#1#2#3#4{\left[
\begin{array}{c}
#1\\
#2\\
#3\\
#4\\
\end{array}
\right]}

%\newtheorem{proposition}{Proposition}
%\newtheorem{corollary}{Corollary}
%\newtheorem{theorem}{Theorem}
%\newtheorem{lemma}{Lemma}

\newcommand{\G} {\Gamma}
\newcommand{\Gin} {\Gamma_{in}}
\newcommand{\Gout} {\Gamma_{out}}
\newcommand{\insub}{{\rm in}}
\newcommand{\outsub}{{\rm out}}

\newtheorem{remark}{Remark}

\title{Notes on preconditioning for implicit CD systems}
\begin{document}

\maketitle
Todos
\begin{enumerate}
\item Document current AGMG verification ($h$ and $N$ independence)
\item Compare AGMG/Wathen downwind GS
\item Analyze static condensation case
\item Implement $A+A^T$ AGMG coarsening
\end{enumerate}

\section{Introduction}

A common approach to the numerical solution of the Navier-Stokes equations
\begin{align*}
\pd{u}{t} + u\cdot\grad u + \grad p &= f \\
\div u &= 0
\end{align*}
is to employ a splitting or projection method, where we 
\begin{enumerate}
\item compute an intermediate velocity $u^*$ by solving $$\frac{(u^*-u^k)}{dt} - \nu\Delta u^* = f(t^k)$$
\item correct the velocity by solving 
$$\begin{cases}
\frac{(u^{k+1}-u^*)}{dt} + \nabla p^k &= 0\\
\nabla \cdot u^{k+1} &= 0\\
u^{k+1}\cdot n &= 0 \quad \text{on the boundary}
\end{cases}$$
\end{enumerate}
This above step is the reason splitting schemes are referred to as projection methods - the weak form of (2) can be interpreted as the scaled $L^2$ projection of $u^*$ onto a divergence-free $u^{k+1}$ (after multiplying by a test function, $p$ can then be viewed as a Lagrange multiplier to enforce the divergence free constraint).  Since pressure is undetermined in the second step, we need to produce an approximation for $p$.  Based on the assumption that $\nabla \cdot u^{k+1}$ must be $0$, we can take the divergence of the second equation to get $$\nabla \cdot \frac{u^*}{dt} + \nabla\cdot \nabla p^k = 0,$$ which yields a Poisson equation for $p$ given $u^*$. 

The overall scheme is then 
\begin{enumerate}
\item Solve for $u^*$ through $$\frac{(u^*-u^k)}{dt} - \nu\Delta u^* = f(t^k)$$
\item  Solve for $p$ through $$\Delta p = \frac{1}{dt}\nabla \cdot u^*$$
\item Solve for $u^{k+1}$ through $$\frac{(u^{k+1}-u^*)}{dt} + \nabla p^k = 0.$$
\end{enumerate}
Note that both steps 2 and 3 involve the solution of a Poisson and $L^2$-projection problem, respectively, both of which involve symmetric, positive-definite operators which can be solved very efficiently in a scalable fashion.  However, as step 1 involves the solution of a non-symmetric convection-diffusion system, the full system is eschewed in favor of 

A particularly restrictive aspect of this solution method is the CFL condition for spectral finite elements
\[
dt < O\LRp{\LRb{u}\frac{h}{N^2}},
\]
which results from the $O(1/N^2)$ minimum spacing between the GLL nodes of order $N$ on a given element.  For high $N$ ($N>20$ and higher), the CFL condition can be incredibly restrictive 

\section{Scalar convection-diffusion}

We investigate first the scalar convection-diffusion equation with homogeneous Dirichlet boundary condtions.  
\begin{align*}
\div(\beta u) - \epsilon \Delta u &= f\\
\left.u\right|_{\Gamma} = 0,
\end{align*}
and preconditioning strategies for systems resulting from high order spectral element discretizations of this equation.  

\subsection{AGMG}

\subsubsection{Static condensation}

Static condensation involves the representation of interior degrees of freedom (nodes in the interior of an element) in terms of globally coupled degrees of freedom through the local Schur complement.  

Todos
\begin{enumerate}
\item Try iterative solver on single element system - how do you do matrix-free condensation?  
\item Balance cost - what is the baseline?  Can we do a GPU workgroup implementation?  
\end{enumerate}

\subsection{``Downwind'' preconditioner}


\bibliographystyle{unsrt}
\bibliography{paper}

\end{document}